%\documentclass[aps,prl,twocolumn,showpacs,superscriptaddress,groupedaddress]{revtex4}  % for review and submission
\documentclass[aps,preprint,showpacs,superscriptaddress,groupedaddress]{revtex4}  % for double-spaced preprint
\usepackage{dcolumn}   % needed for some tables
\usepackage{bm}        % for math
\usepackage{amssymb}   % for math
\usepackage{amsfonts}  % for math
\usepackage{amsmath}   % for math
\usepackage{mathcomp}  % for math
\usepackage{graphicx}  % for figures
\usepackage{subfigure} % for figures
\usepackage{multirow}  % for tables
\usepackage{hyperref}  % for urls

\usepackage{verbatim}
\usepackage{color}
%\usepackage{ulem}

\newcommand{\awk}[1]{{\color{red} awkward $\rightarrow$}{\em #1}}
\newcommand{\note}[1]{{\color{blue} $\ll$ #1 $\gg$}}
\newcommand{\edit}[2]{{\color{blue} #1 }{\sout{#2}}}
\newcommand{\add}[1]{{\color{blue} #1}}

% avoids incorrect hyphenation, added Nov/08 by SSR
\hyphenation{ALPGEN}
\hyphenation{EVTGEN}
\hyphenation{PYTHIA}

\begin{document}

\title{Properties of Permutations Sectors for the Free Fermi Gas}

% authors
\author{Ethan W. Brown}
\email{brown122@illinois.edu}
\affiliation{Department of Physics, University of Illinois at Urbana-Champaign, 1110 W.\ Green St.\ , Urbana, IL  61801-3080, USA}
\affiliation{Lawrence Livermore National Lab, 7000 East Ave, L-415, Livermore CA 94550, USA}
\date{\today}

\begin{abstract}
\end{abstract}

\pacs{}
\maketitle

\section{Separating Permutation Space}

We begin by writing the partition function of $N$ particles as

\begin{eqnarray}
  \mathcal{Z}^{(N)} &=& \frac{1}{N!} \int d^{D}x^{(1)} \dots d^{D}x^{(N)} (x^{(1)},\dots,x^{(N)};\hbar \beta \mid x^{(1)},\dots,x^{(N)};0) \\
                    &=& \frac{1}{N!} \int d^{D}x^{(1)} \dots d^{D}x^{(N)} \sum_{p(\nu)} \epsilon_{p(\nu)} (x^{(p(1))},\dots,x^{(p(N))};\hbar \beta \mid x^{(1)},\dots,x^{(N)};0)
\end{eqnarray}

where the sum is over all possible permutations, each with sign $\epsilon_{p(\nu)}$. Generally we write,

\begin{equation}
  (x_{b}^{(1)},\dots,x_{b}^{(N)};\hbar \beta \mid x_{a}^{(1)},\dots,x_{a}^{(N)};0) = \prod_{\nu=1}^{N} [\int \mathcal{D}^{D}x^{\nu}\exp^{-\mathcal{A}^{(N)}/\hbar}]
\end{equation}

where the $N$-particle action is given by,

\begin{equation}
  \mathcal{A}^{(N)} = \int_{0}^{\hbar \beta} d\tau [\sum_{\nu=1}^{N}(\frac{M^{(\nu)}}{2}(\dot{x}^{(\nu)})^{2} - V(x^{(\nu)})) - \frac{1}{2}\sum_{\nu\neq\nu'=1}^{N}V_{int}(x^{(\nu)}-x^{(\nu')})]
\end{equation}

For free particles, $V, V_{int} = 0$. Thus,

\begin{eqnarray}
  (x_{b}^{(p(1))},\dots,x_{b}^{(p(N))};\hbar \beta \mid x_{a}^{(1)},\dots,x_{a}^{(N)};0) &=& \prod_{\nu=1}^{N} (x_{b}^{(p(\nu))};\hbar \beta \mid x_{a}^{(\nu)};0) \\
                                                                                         &=& \prod_{\nu=1}^{N} [\int_{x^{(\nu)}(0)}^{x^{(p(\nu))}\hbar\beta} \mathcal{D}^{D}x^{\nu}\exp^{-\mathcal{A_{\nu}}^{(N)}/\hbar}] \\
                                                                                         &=& \prod_{\nu=1}^{N} [\int_{x^{(\nu)}(0)}^{x^{(p(\nu))}\hbar\beta} \mathcal{D}^{D}x^{\nu}\exp{[-\frac{1}{\hbar} \int_{0}^{\hbar\beta}d\tau \frac{M^{(\nu)} (\dot{x}^{(\nu)})^{2}}{2}]} \\
                                                                                         &=& \prod_{\nu=1}^{N} \frac{1}{\sqrt{2\pi\hbar^{2}\beta/M^{(\nu)}}^{D}} \exp{[-\frac{M^{(\nu)}}{2\hbar^{2}\beta} (x_{b}^{(p(\nu))} - x_{a}^{(\nu)})^{2}]}
\end{eqnarray}

Defining $\lambda \equiv \frac{\hbar^{2}}{2M^{(\nu)}}$,

\begin{equation}
  \mathcal{Z}^{(N)}_{0} = \frac{1}{\sqrt{4\pi\lambda\beta}^{ND}} \frac{1}{N!} \int d^{D}x^{(1)} \dots d^{D}x^{(N)} \sum_{p(\nu)} \epsilon_{p(\nu)} \prod_{\nu=1}^{N} \exp{[-\frac{(x^{(p(\nu))}-x^{(\nu)})^{2}}{4\lambda\beta}]}
\end{equation}

Now all paths decompose into mutually disconnected groups, each with a winding number $\omega$. In general,

\begin{equation}
  \mathcal{Z}^{(N)}_{0,\omega} = \mathcal{Z}_{0}(\omega\beta) = \frac{V_{D}}{\sqrt{4\pi\lambda\omega\beta}^{D}}
\end{equation}

The number of elements consisting of $C_{1} \dots C_{N}$ cycles of length $\omega_{1} \dots \omega_{N}$ is,

\begin{equation}
  M(C_{1} \dots C_{N}) = \frac{N!}{\prod_{\omega=1}^{N} C_{\omega}! \omega^{C_{\omega}}}
\end{equation}

Thus,

\begin{equation}
  \mathcal{Z}_{0}^{(N)}(\beta) = \frac{1}{N!} \sum_{p(\nu)} \epsilon_{p(\nu)} M(C_{1} \dots C_{N}) \prod_{\omega=1, N=\sum_{\omega}\omega C_{\omega}}^{N} \mathcal{Z}_{0}(\omega\beta)^{C_{\omega}}
\end{equation}

Using this and doing some reordering, we find

\begin{eqnarray}
  \mathcal{Z}_{0}^{(N)}(\beta) &=& \sum_{C_{1} \dots C_{N}, N=\sum_{\omega}\omega C_{\omega}} \frac{M(C_{1} \dots C_{N}) \prod_{\omega=1}^{N} (\pm 1)^{C_{\omega}(\omega-1)}}{N!} \prod_{\omega=1}^{N} \mathcal{Z}_{0}(\omega\beta)^{C_{\omega}} \\
                               &=& \sum_{\mathcal{P}} \tilde{M}(C_{1} \dots C_{N}) \prod_{\omega=1}^{N} \mathcal{Z}_{0}(\omega\beta)^{C_{\omega}} \\
                               &=& \sum_{\mathcal{P}} \prod_{\omega=1}^{N} \frac{1}{C_{\omega}!} [(\pm 1)^{\omega-1} \frac{\mathcal{Z}_{0}(\omega\beta)}{\omega}]^{C_{\omega}}
\end{eqnarray}

where we relabeled $C_{1} \dots C_{N}, N=\sum_{\omega}\omega C_{\omega}$ as $\mathcal{P}$.


\section{Sector Probabilities}

From the above decomposition of the partition function, we can define the probability of a given permutation section $\mathcal{P}$ as,

\begin{eqnarray}
  P_{\mathcal{P}}^{(N)}(\beta) &=& \frac{1}{\mathcal{Z}_{0}^{(N)}(\beta)} \tilde{M}(\mathcal{P}) \prod_{\omega=1}^{N} \mathcal{Z}_{0}(\omega\beta)^{C_{\omega}} \\
                               &=& \frac{1}{\sum_{\mathcal{P}} \tilde{M}(\mathcal{P}) \prod_{\gamma=1}^{N} \mathcal{Z}_{0}(\gamma\beta)^{C_{\gamma}}} \tilde{M}(\mathcal{P}) \prod_{\omega=1}^{N} \mathcal{Z}_{0}(\omega\beta)^{C_{\omega}} \\
                               &=& \tilde{M}(\mathcal{P}) \prod_{\omega=1}^{N} [\frac{\mathcal{Z}_{0}(\omega\beta)}{(\sum_{\mathcal{P}} \tilde{M}(\mathcal{P}) \prod_{\gamma=1}^{N} \mathcal{Z}_{0}(\gamma\beta)^{C_{\gamma}})^{1/C_{\omega}}}]^{C_{\omega}} \\
                               &=& \tilde{M}(\mathcal{P}) \prod_{\omega=1}^{N} P_{\omega}(\beta)^{C_{\omega}}
\end{eqnarray}

where we have defined $P_{\omega}(\beta)$ as the probability of a cycle of length $\omega$. What's important is that this probability is indepedent of sector $\mathcal{P}$, implying that (at least for the free gas), the entire permutation space can reconstructed from $N$ parameters and known combinatorial factors.

Going one step further, we want to show that each of these $N$ parameters can be reconstructed from a single value, e.g. $P_{\omega=2}(\beta)$. From the above equation, we can see it is equivalent to show that $\mathcal{Z}_{0}(\omega\beta)$ can be written as a function of $\mathcal{Z}_{0}(\gamma\beta)$ with $\omega \neq \gamma$:

\begin{equation}
  \mathcal{Z}_{0}(\omega\beta) = \frac{V_{D}}{\sqrt{4\pi\lambda\omega\beta}^{D}} = (\frac{\gamma}{\omega})^{D/2} \mathcal{Z}_{0}(\gamma\beta)
\end{equation}

Thus,

\begin{equation}
  P_{\omega}(\beta) = \frac{\mathcal{Z}_{0}(\omega\beta)}{\mathcal{Z}_{0}^{(N)}(\beta)} = \frac{(\frac{\gamma}{\omega})^{D/2} \mathcal{Z}_{0}(\gamma\beta)}{\mathcal{Z}_{0}^{(N)}(\beta)} = (\frac{\gamma}{\omega})^{D/2} P_{\gamma}(\beta).
\end{equation}

Finally then we see that

\begin{equation}
  P_{\mathcal{P}}^{(N)}(\beta) = \tilde{M}(\mathcal{P}) \prod_{\omega=1}^{N} [(\frac{\gamma}{\omega})^{D/2} P_{\gamma}(\beta)]^{C_{\omega}}
\end{equation}

for any value of $\gamma = 1 \dots N$. Therefore we can now reconstruct the entire permutation space from a single parameter and known combinatorial factors.


\section{Energy per Sector}

Recall the energy is defined as,

\begin{equation}
  E^{(N)}(\beta) = \frac{\partial}{\partial\beta} \log{\mathcal{Z}^{(N)}(\beta)} = \frac{1}{\mathcal{Z}^{(N)}(\beta)} \frac{\partial}{\partial\beta} \mathcal{Z}^{(N)}(\beta).
\end{equation}

For the free gas then,

\begin{eqnarray}
  E_{0}^{(N)}(\beta) &=& \frac{1}{\mathcal{Z}_{0}^{(N)}(\beta)} \frac{\partial}{\partial\beta} \mathcal{Z}_{0}^{(N)}(\beta) \\
                     &=& \frac{1}{\mathcal{Z}_{0}^{(N)}(\beta)} \frac{\partial}{\partial\beta} \sum_{\mathcal{P}} \tilde{M}(\mathcal{P}) \prod_{\omega=1}^{N} \mathcal{Z}_{0}(\omega\beta)^{C_{\omega}} \\
                     &=& \frac{1}{\mathcal{Z}_{0}^{(N)}(\beta)} \sum_{\mathcal{P}} \tilde{M}(\mathcal{P}) \frac{\partial}{\partial\beta} \prod_{\omega=1}^{N} \mathcal{Z}_{0}(\omega\beta)^{C_{\omega}} \\
                     &=& \frac{1}{\mathcal{Z}_{0}^{(N)}(\beta)} \sum_{\mathcal{P}} \tilde{M}(\mathcal{P}) \sum_{\gamma=1}^{N} C_{\gamma} \frac{\frac{\partial}{\partial\beta}\mathcal{Z}_{0}(\gamma\beta)}{\mathcal{Z}_{0}(\gamma\beta)} \prod_{\omega=1}^{N} \mathcal{Z}_{0}(\omega\beta)^{C_{\omega}} \\
                     &=& \frac{1}{\mathcal{Z}_{0}^{(N)}(\beta)} \sum_{\mathcal{P}} \tilde{M}(\mathcal{P}) \prod_{\omega=1}^{N} \mathcal{Z}_{0}(\omega\beta)^{C_{\omega}} \sum_{\gamma=1}^{N} C_{\gamma} E_{0}(\gamma\beta) \\
                     &=& \sum_{\mathcal{P}} P_{\mathcal{P}}^{(N)}(\beta) \sum_{\gamma=1}^{N} C_{\gamma} E_{0}(\gamma\beta)
\end{eqnarray}

\bibliography{FFGPermSectors}{}

%\pagebreak
%\input{supplementary.tex}

\end{document}
